%% This paper describes the results of the Machine Learning practical
%% project. It should be approximately 8 pages long and contain the
%% following sections:
%%
%% 1) Title, authors, abstract
%% 2) Introduction
%% 3) Application
%% 4) Methods
%% 5) Experimental Results
%% 6) Discussion / Conclusion
%% References
\documentclass[preprint,journal]{vgtc}
\let\ifpdf\relax
%% Uncomment one of the lines above depending on where your paper is
%% in the conference process. ``review'' and ``widereview'' are for review
%% submission, ``preprint'' is for pre-publication, and the final version
%% doesn't use a specific qualifier. Further, ``electronic'' includes
%% hyperreferences for more convenient online viewing.

%% Please use one of the ``review'' options in combination with the
%% assigned online id (see below) ONLY if your paper uses a double blind
%% review process. Some conferences, like IEEE Vis and InfoVis, have NOT
%% in the past.

%% Please note that the use of figures is not permitted on the first page
%% of the journal version.  Figures should begin on the second page and be
%% in CMYK or Grey scale format, otherwise, colour shifting may occur
%% during the printing process.  Papers submitted with figures on the
%% first page will be refused.

%% These three lines bring in essential packages: ``mathptmx'' for Type 1
%% typefaces, ``graphicx'' for inclusion of EPS figures. and ``times''
%% for proper handling of the times font family.

\usepackage{mathptmx}
\usepackage{graphicx,xcolor,booktabs, tabularx,amsmath}
\usepackage{times}
\newcommand{\todo}[1]{\textbf{\textcolor{blue}{Todo: #1 }}}
%% We encourage the use of mathptmx for consistent usage of times font
%% throughout the proceedings. However, if you encounter conflicts
%% with other math-related packages, you may want to disable it.

%% If you are submitting a paper to a conference for review with a double
%% blind reviewing process, please replace the value ``0'' below with your
%% OnlineID. Otherwise, you may safely leave it at ``0''.
\onlineid{0}

%% declare the category of your paper, only shown in review mode
\vgtccategory{}

%% allow for this line if you want the electronic option to work properly
%\vgtcinsertpkg

%% In preprint mode you may define your own headline.
\preprinttext{}

%% Paper title.

\title{Bayes network classifiers for fMRI}

%% This is how authors are specified in the journal style

%% indicate IEEE Member or Student Member in form indicated below
\author{Mattijs Meiboom, Robin Mills, Tijn Schouten, Spyros Ioakeimidis}
\authorfooter{
\item
  Mattijs Meiboom is with University of Groningen, E-mail: m.meiboom@student.rug.nl.
\item
  Robin Mills is with University of Groningen, E-mail: .
\item
  Tijn Schouten is University of Groningen, E-mail: t.m.schouten.2@student.rug.nl.
\item
  Spyros Ioakeimidis  is with University of Groningen, E-mail: spyrosikmd@gmail.com.
}



%other entries to be set up for journal
%\shortauthortitle{}


%% Abstract section.
\abstract{
Functional magnetic resonance imaging (fMRI) is a brain mapping technique that allows researchers to measure activity in a subject's brain. This technique is typically used to determine which areas of the brain are involved in performing a certain task. Using this knowledge, we may also be able to determine what task was performed by looking at fMRI data. In this paper, we will discuss the use of machine learning algorithms for the classification of fMRI data. By using classifiers we will try to predict what stimulus a subject is looking at, based on the brain activity during that time.
} % end of abstract

%% Keywords that describe your work. Will show as 'Index Terms' in journal
%% please capitalize first letter and insert punctuation after last keyword
\keywords{fMRI, Machine learning, Classification.}

%% Copyright space is enabled by default as required by guidelines.
%% It is disabled by the 'review' option or via the following command:
\nocopyrightspace


\begin{document}

%% The ``\maketitle'' command must be the first command after the
%% ``\begin{document}'' command. It prepares and prints the title block.
%% the only exception to this rule is the \firstsection command
\firstsection{Introduction}
\maketitle
%% Now the actual contents of first section
- basic terms introduced (fmri, classification, ...)
- related work and results
- state of the art: multivariate pattern analysis (\cite{Haxby2012852})
- structure of rest of the paper
\section{Application}
- prediction
- cognitive state
\section{Methods}
- supervised learning
- naive bayes
- multivariate
- pca
- comparison
- classifying subtrials vs trials (ie 54 vs 27)

\subsection{Data acquisition and dimensions}
The dataset used for our experiments was originally collected by Marcel Just et al. from Carnegie Mellon University's Center for Cognitive Brain Imaging and published as the StarPlus fMRI dataset. This set consists of fMRI data for 6 subjects that was obtained while having each subject participate in a number of trials.

In these trials, subjects were presented with a stimulus on a computer screen. This stimulus was either a picture or a sentence. After a period of time, the stimulus would be replaced with a stimulus of the opposite type (i.e. picture with sentence and vice versa) after which the subject would indicate by pressing a button whether the sentence correctly described the picture. At the end of each trial, approximately 54 fMRI scans have be taken with an interval of 500ms.
The entire dataset consist of voxel data and metadata providing information about the dataset. Included in this information is not only the trial and scan each voxel belongs to, but also the type of the trial (picture first or sentence first). This makes the dataset suitable for classification using supervised learning.

In total, we have data available for approximately 4000 voxels per scan, 54 scans per trial, 40 trials per subject, 6 subjects.

\section{Experimental Results}
\section{Discussion / Conclusion}

\bibliographystyle{abbrv}
%%use following if all content of bibtex file should be shown
\nocite{*}
\bibliography{references}
\end{document}