\documentclass[a4paper, 11pt]{scrartcl}
\usepackage[utf8]{inputenc} 
\usepackage[T1]{fontenc}  
\usepackage[left=2.5cm, right=2.5cm, top=2.5cm, bottom=3cm, a4paper]{geometry}
\usepackage[pdftex]{graphics} 
\usepackage{mathptmx}
\usepackage{graphicx}



\title{\Large Out title here}
\author{\small Mattijs Meiboom (s1398342), Robin Mills (student number)\\
			\small Tijn Schouten (s1836560), Spyros Ioakeimidis (2144603)}
\date{\small \today}

\setlength{\parindent}{0pt} 

\begin{document}

\maketitle

\thispagestyle{empty}

\section{Introduction}

This project is about classifying fMRI data in order to detect the cognitive states associated with viewing different stimuli. Specifically, viewing sentences versus viewing pictures will evoke different patterns of activity in the brain. By using classifiers we will try to capture these patterns by predicting which stimuli a participant is looking at, based on the brain activity during that time.

\section{Application}

Explain how you get data or simulator, how many examples, how many inputs, outputs, whether it is an RL, classification, or regression problem.

\section{Methods}

Explain the methods that you will compare. Identify the learning algorithm and the optimisation method.

\section{Setup of Experiments}

\section{Programming Language}

The chosen programming language is Python. We have chosen Python because ...

\section{Planning}

Describe the different phases and the planning deadlines (until November)

\end{document}
