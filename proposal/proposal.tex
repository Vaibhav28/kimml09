\documentclass[a4paper, 11pt]{scrartcl}
\usepackage[utf8]{inputenc} 
\usepackage[T1]{fontenc}  
\usepackage[left=2.5cm, right=2.5cm, top=2.5cm, bottom=3cm, a4paper]{geometry}
\usepackage[pdftex]{graphics} 
\usepackage{mathptmx}
\usepackage{graphicx}



\title{\Large Bayes network classifiers for fRMI}
\author{\small Mattijs Meiboom (s1398342), Robin Mills (s1544179)\\
			\small Tijn Schouten (s1836560), Spyros Ioakeimidis (s2144603)}
\date{\small \today}

\setlength{\parindent}{0pt} 

\begin{document}

\maketitle

\thispagestyle{empty}

\section{Introduction}

This project is about classifying fMRI data in order to detect the cognitive states associated with viewing different stimuli. Specifically, viewing sentences versus viewing pictures will evoke different patterns of activity in the brain. By using classifiers we will try to capture these patterns by predicting which stimuli a participant is looking at, based on the brain activity during that time.

\section{Application}

Explain how you get data or simulator, how many examples, how many inputs, outputs, whether it is an RL, classification, or regression problem.

\section{Methods}

Explain the methods that you will compare. Identify the learning algorithm and the optimisation method.

\section{Setup of Experiments}

\section{Programming Language}
\label{sec:language}

We have chosen to implement our classification algorithms using Python. To be specific,  the well-known SciPy library. SciPy is open-source software for mathematics, science, and engineering and offers similar functionality compared to MatLab. Even though our datasets have been stored in MatLab's \textit{.mat} format, SciPy can work with this as well.

The main motivation for choosing Python over MatLab is our familiarity with Python, its performance and its clean coding style. As an added benefit, anyone is able to reproduce our classification results even if they don't own a MatLab license. 

Note that in the end, the specific choice is trivial with respect to our results. Both MatLab and Python (with some added libraries) support similar functionality.

\section{Planning}

Describe the different phases and the planning deadlines (until November)

\begin{description}
	\item[11th - 21st September]\-\\
	Research ...
	\item[22nd - 28th September]\-\\
	Model definition ...
	\item[29th - 19th October]\-\\
	Implementation and Parameter estimation ...
	\item[20th - 22nd October]\-\\
	Preparing the presentation ...
	\item[28th October]\-\\
	Final delivery of the document and the implementation
\end{description}

\end{document}
