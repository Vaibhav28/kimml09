\documentclass[a4paper, 11pt]{scrartcl}
\usepackage[utf8]{inputenc} 
\usepackage[T1]{fontenc}  
\usepackage[left=2.5cm, right=2.5cm, top=2.5cm, bottom=3cm, a4paper]{geometry}
\usepackage{hyperref}
\usepackage[pdftex]{graphics} 
\usepackage{mathptmx}
\usepackage{graphicx}

\title{\Large Bayes network classifiers for fMRI}
\author{\small Mattijs Meiboom (s1398342), Robin Mills (s1544179)\\
			\small Tijn Schouten (s1836560), Spyros Ioakeimidis (s2144603)}
\date{\small \today}

\setlength{\parindent}{0pt} 

\begin{document}

\maketitle

\thispagestyle{empty}

\section{Introduction}

This project is about classifying fMRI data in order to detect the cognitive states of subjects associated with viewing different stimuli. Specifically, viewing sentences versus viewing pictures will evoke different patterns of activity in the brain. By using classifiers we will try to capture these patterns by predicting which stimuli a participant is looking at, based on the brain activity during that time.

\section{Application}
\label{sec:application}

For our project, we are using a collection of data from the StarPlus project. This dataset contains the results from a trial experiment performed at Carnegie Mellon University. The data was originally collected by Marcel Just \textit{et al} \cite{StarPlus:2004kx}.

The dataset contains data on 6 subjects that each completed a number of trials. For each trial (cf. Section~\ref{sec:setup}), a brain scan was made every 500 ms, totalling to approximately 54 scans per trial. Each scan consists of a list of approximately 5000 voxels (three dimensional pixels), of predefined regions of the brain. Along with each trial, metadata is available on the type of trial (e.g. what stimulus was shown), the result of the trial, and to which subject the scan belongs.

Since we will be trying to classify brain patterns to determine what stimulus was active when the scan was taken, our application deals with a typical classification problem.

\section{Methods}
\label{sec:methods}

The dataset we use originated from a project that researched classifying cognitive state using Gaussian Naive Bayes classifiers. This resulted in an accuracy of approximately 85\%. According to literature, a higher accuracy could be obtained by using a so-called \textit{Tree Augmented Naive Bayesian network classifier}, or TAN \cite{Friedman:1997gw}. We will use this classification method to investigate whether we are indeed able to reach a higher accuracy.

To verify our results, we will use the original Naive Bayesian classifier as a benchmark and compare it to our TAN results.

\section{Setup of Experiments}
\label{sec:setup}

Participants performed an experiment while laying in an fMRI scanner. The experiment consisted of a sets of trials where participants were presented a first stimulus for four seconds, followed by a blank screen lasting four seconds, and then a second stimulus for four seconds, followed by a rest period of 15 seconds. For the first half of the trials, a picture was presented as the first stimulus and a sentence as the second stimulus, while for the second half of trials, a sentence was presented as the first stimulus and a picture as the second. Participants had four seconds after the second stimulus appeared to indicate that the sentence and picture belonged together, after which the stimulus disappeared. If participants did not press a button the stimulus would disappear after four seconds.

\section{Programming Language}
\label{sec:language}

We have chosen to implement our classification algorithms using Python. To be specific,  the well-known SciPy library. SciPy is open-source software for mathematics, science, and engineering and offers similar functionality compared to MatLab. Even though our datasets have been stored in MatLab's \textit{.mat} format, SciPy can work with this as well.

The main motivation for choosing Python over MatLab is our familiarity with Python, its performance and its clean coding style. As an added benefit, anyone is able to reproduce our classification results even if they don't own a MatLab license. 

Note that in the end, the specific choice is trivial with respect to our results. Both MatLab and Python (with some added libraries) support similar functionality.

\section{Planning}

In this section we describe the various phases and the planning deadlines that the group will follow.

\begin{description}
	\item[11th - 21st September]\-\\
	During this time period, the group members will conduct an initial research. The research will focus on the Bayes network classifiers and especially on the TAN tree algorithm.
	\item[22nd - 28th September]\-\\
	During this time period, the definition of the various models will be realised.
	\item[29th - 19th October]\-\\
	Implementation and Parameter estimation ...
	\item[20th - 22nd October]\-\\
	During this time period the group members will prepare the presentation of the project.
	\item[28th October]\-\\
	Until the aforementioned date the final version of the document and the implementation will be delivered.
\end{description}


\nocite{*} %DO NOT UNCOMMENT, only for test purpose
\bibliographystyle{IEEEtran}
\bibliography{bibliography}

\end{document}
